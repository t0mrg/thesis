% bristolthesis template.tex file
% Best not to fiddle with this much/at all or things might break
% This needs to line up with the contents of bristolthesis.cls and vice versa
%
%
%
%
%

\documentclass[11pt,twoside]{bristolthesis}

%% Packages - these are those which i used for my thesis so might not be specific to yours
\usepackage{graphicx,latexsym}
\usepackage{amsmath}
\usepackage{amssymb}
\usepackage{amsthm}
\usepackage{longtable}
\usepackage{booktabs}
\usepackage{setspace}
\usepackage{siunitx}
% \usepackage{chemarr} %% Useful for one reaction arrow, useless if you're not a chem major
\usepackage[hyphens]{url}
\usepackage{hyperref}
\usepackage{lmodern}
\usepackage{float}
\floatplacement{figure}{H}
\usepackage{rotating}
% \usepackage{times} % other fonts are available like times, bookman, charter, palatino

%% Paramaters for global document
\hypersetup{colorlinks = false}
\renewcommand{\UrlBreaks}{\do\/\do\a\do\b\do\c\do\d\do\e\do\f\do\g\do\h\do\i\do\j\do\k\do\l\do\m\do\n\do\o\do\p\do\q\do\r\do\s\do\t\do\u\do\v\do\w\do\x\do\y\do\z\do\A\do\B\do\C\do\D\do\E\do\F\do\G\do\H\do\I\do\J\do\K\do\L\do\M\do\N\do\O\do\P\do\Q\do\R\do\S\do\T\do\U\do\V\do\W\do\X\do\Y\do\Z\do\0\do\1\do\2\do\3\do\4\do\5\do\6\do\7\do\8\do\9\do\%\do\.\do\-}
\renewcommand{\chapterautorefname}{Chapter}
\usepackage{times} % other fonts are available like times, bookman, charter, palatino
\usepackage{caption}

% Use ref for internal links
\renewcommand{\hyperref}[2][???]{\autoref{#1}}
\def\chapterautorefname{Chapter}
\def\sectionautorefname{Section}
\def\subsectionautorefname{Subsection}
\usepackage{caption}
\captionsetup{width=5in}

% Syntax highlighting #22
%%%

%%% YAML header functions
\title{I made this template based on thesisdown to comply with the University of Bristol regulations}
\author{Thomas Battram}
\date{September 2020}
\university{University of Bristol}
\faculty{Health Sciences}
\school{Bristol Medical School}
\group{MRC Integrative Epidemiology Unit}
\wordcount{}
\degree{Population Health Sciences}
\logo{figure/index/UoBcrest.pdf}
%%%


%%% The document formatting
\makeatletter
\def\maxwidth{ %
  \ifdim\Gin@nat@width>\linewidth
    \linewidth
  \else
    \Gin@nat@width
  \fi
}
\makeatother

\renewcommand{\contentsname}{Table of Contents}

\setlength{\parskip}{14truept}

  \setlength{\parskip}{\baselineskip}
  \usepackage[parfill]{parskip}

\providecommand{\tightlist}{%
  \setlength{\itemsep}{0pt}\setlength{\parskip}{0pt}}

\Acknowledgements{
THIS IS WHERE YOU THANK PEOPLE!!!!!!!!!!!!!!!!!!!!!!!!!!!!!!!!!!!!!!!!!!!!!!!!!!!!!!!!!!!!!!!!!
}

\Declaration{
I declare that the work in this dissertation was carried out in accordance with the requirements of the University's Regulations and Code of Practice for Research Degree Programmes and that it has not been submitted for any other academic award. Except where indicated by specific reference in the text, the work is the candidate's own work. Work done in collaboration with, or with the assistance of, others, is indicated as such. Any views expressed in the dissertation are those of the author.

\bigskip
\bigskip
\bigskip
\bigskip
\bigskip

Signed

\bigskip
\bigskip
\bigskip
\bigskip
\bigskip

Dated
}

\Abstract{
My abstract will go here and it will be a solid abstract. Full of the things that go in abstracts. Such as numbers, acronyms, other words, and lots of punctuation.

It will have multiple paragraphs too!
}

\Abbreviations{
\textbf{ACR} - acronym

\textbf{aACR} - another acronym
}

	\usepackage{tikz} \usepackage{booktabs} \usepackage{longtable} \usepackage{siunitx} \pagestyle{plain}
	\usepackage{booktabs}
 \usepackage{longtable}
 \usepackage{array}
 \usepackage{multirow}
 \usepackage{wrapfig}
 \usepackage{float}
 \usepackage{colortbl}
 \usepackage{pdflscape}
 \usepackage{tabu}
 \usepackage{threeparttable}
 \usepackage{threeparttablex}
 \usepackage[normalem]{ulem}
 \usepackage{makecell}
 \usepackage{xcolor}


\newlength{\cslhangindent}
\setlength{\cslhangindent}{1.5em}
\newenvironment{cslreferences}%
  {\setlength{\parindent}{0pt}%
  \everypar{\setlength{\hangindent}{\cslhangindent}}\ignorespaces}%
  {\par}


%%% Main document
\spacing{1}
\begin{document}
  \maketitle

\frontmatter % this stuff will be roman-numbered
\pagestyle{empty} % this removes page numbers from the frontmatter
  \begin{abstract}
    My abstract will go here and it will be a solid abstract. Full of the things that go in abstracts. Such as numbers, acronyms, other words, and lots of punctuation.

    It will have multiple paragraphs too!
  \end{abstract}
  \begin{acknowledgements}
    THIS IS WHERE YOU THANK PEOPLE!!!!!!!!!!!!!!!!!!!!!!!!!!!!!!!!!!!!!!!!!!!!!!!!!!!!!!!!!!!!!!!!!
  \end{acknowledgements}
  \begin{declaration}
    I declare that the work in this dissertation was carried out in accordance with the requirements of the University's Regulations and Code of Practice for Research Degree Programmes and that it has not been submitted for any other academic award. Except where indicated by specific reference in the text, the work is the candidate's own work. Work done in collaboration with, or with the assistance of, others, is indicated as such. Any views expressed in the dissertation are those of the author.

    \bigskip
    \bigskip
    \bigskip
    \bigskip
    \bigskip

    Signed

    \bigskip
    \bigskip
    \bigskip
    \bigskip
    \bigskip

    Dated
  \end{declaration}
  \hypersetup{linkcolor=black}
  \setcounter{tocdepth}{3}
  \tableofcontents
  \listoftables
  \listoffigures

\spacing{1.5}
\mainmatter % here the regular arabic numbering starts
\pagestyle{plain}
\hypertarget{preface}{%
\chapter*{Preface}\label{preface}}
\addcontentsline{toc}{chapter}{Preface}

This template is based on (and in many places copied directly from) the Reed College LaTeX template, but hopefully it will provide a nicer interface for those that have never used TeX or LaTeX before. Using \emph{R Markdown} will also allow you to easily keep track of your analyses in \textbf{R} chunks of code, with the resulting plots and output included as well. The hope is this \emph{R Markdown} template gets you in the habit of doing reproducible research, which benefits you long-term as a researcher, but also will greatly help anyone that is trying to reproduce or build onto your results down the road.

Hopefully, you won't have much of a learning period to go through and you will reap the benefits of a nicely formatted thesis. The use of LaTeX in combination with \emph{Markdown} is more consistent than the output of a word processor, much less prone to corruption or crashing, and the resulting file is smaller than a Word file. While you may have never had problems using Word in the past, your thesis is likely going to be about twice as large and complex as anything you've written before, taxing Word's capabilities. After working with \emph{Markdown} and \textbf{R} together for a few weeks, we are confident this will be your reporting style of choice going forward.

\textbf{Why use it?}

\emph{R Markdown} creates a simple and straightforward way to interface with the beauty of LaTeX. Packages have been written in \textbf{R} to work directly with LaTeX to produce nicely formatting tables and paragraphs. In addition to creating a user friendly interface to LaTeX, \emph{R Markdown} also allows you to read in your data, to analyze it and to visualize it using \textbf{R} functions, and also to provide the documentation and commentary on the results of your project. Further, it allows for \textbf{R} results to be passed inline to the commentary of your results. You'll see more on this later.

\textbf{Who should use it?}

Anyone who needs to use data analysis, math, tables, a lot of figures, complex cross-references, or who just cares about the final appearance of their document should use \emph{R Markdown}. Of particular use should be anyone in the sciences, but the user-friendly nature of \emph{Markdown} and its ability to keep track of and easily include figures, automatically generate a table of contents, index, references, table of figures, etc. should make it of great benefit to nearly anyone writing a thesis project.

\textbf{For additional help with bookdown}
Please visit \href{https://bookdown.org/yihui/bookdown/}{the free online bookdown reference guide}.

\hypertarget{introduction}{%
\chapter{Introduction}\label{introduction}}

\hypertarget{methods}{%
\chapter{Methods}\label{methods}}

\hypertarget{ewas-catalog}{%
\chapter{EWAS Catalog}\label{ewas-catalog}}

\hypertarget{properties-of-ewas}{%
\chapter{Properties of EWAS}\label{properties-of-ewas}}

Here is a reference to Caroline's paper: (Relton \& Davey Smith, 2010)

\hypertarget{m2}{%
\chapter{m2}\label{m2}}

\hypertarget{ewas-gwas-comparison}{%
\chapter{EWAS-GWAS comparison}\label{ewas-gwas-comparison}}

\hypertarget{dnam-lung-cancer-mr}{%
\chapter{DNAm-lung cancer MR}\label{dnam-lung-cancer-mr}}

\hypertarget{conclusion}{%
\chapter*{Conclusion}\label{conclusion}}
\addcontentsline{toc}{chapter}{Conclusion}

If we don't want Conclusion to have a chapter number next to it, we can add the \texttt{\{-\}} attribute.

\textbf{More info}

And here's some other random info: the first paragraph after a chapter title or section head \emph{shouldn't be} indented, because indents are to tell the reader that you're starting a new paragraph. Since that's obvious after a chapter or section title, proper typesetting doesn't add an indent there.

\appendix

\hypertarget{the-first-appendix}{%
\chapter{The First Appendix}\label{the-first-appendix}}

This first appendix includes all of the R chunks of code that were hidden throughout the document (using the \texttt{include\ =\ FALSE} chunk tag) to help with readibility and/or setup.

\textbf{In the main Rmd file}

\textbf{In Chapter \ref{ref-labels}:}

\hypertarget{the-second-appendix-for-fun}{%
\chapter{The Second Appendix, for Fun}\label{the-second-appendix-for-fun}}

\backmatter

\hypertarget{references}{%
\chapter*{References}\label{references}}
\addcontentsline{toc}{chapter}{References}

\markboth{References}{References}

\noindent

\setlength{\parindent}{-0.20in}
\setlength{\leftskip}{0.20in}
\setlength{\parskip}{8pt}

\hypertarget{refs}{}
\begin{cslreferences}
\leavevmode\hypertarget{ref-angel2000}{}%
Angel, E. (2000). \emph{Interactive computer graphics : A top-down approach with opengl}. Boston, MA: Addison Wesley Longman.

\leavevmode\hypertarget{ref-angel2001}{}%
Angel, E. (2001a). \emph{Batch-file computer graphics : A bottom-up approach with quicktime}. Boston, MA: Wesley Addison Longman.

\leavevmode\hypertarget{ref-angel2002a}{}%
Angel, E. (2001b). \emph{Test second book by angel}. Boston, MA: Wesley Addison Longman.

\leavevmode\hypertarget{ref-Relton2010}{}%
Relton, C. L., \& Davey Smith, G. (2010). Epigenetic Epidemiology of Common Complex Disease: Prospects for Prediction, Prevention, and Treatment. \emph{PLoS Medicine}, \emph{7}(10), e1000356. \url{http://doi.org/10.1371/journal.pmed.1000356}
\end{cslreferences}
  \begin{abbreviations}
    \textbf{ACR} - acronym

    \textbf{aACR} - another acronym
  \end{abbreviations}
\end{document}

